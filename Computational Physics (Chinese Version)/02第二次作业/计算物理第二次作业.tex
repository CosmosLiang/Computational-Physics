\documentclass[twoside,twocolumn]{article}
\usepackage{amsmath}
\usepackage[hmarginratio=1:1,top=32mm,columnsep=20pt]{geometry}
\usepackage[hang, small,labelfont=bf,up,textfont=it,up]{caption} 
\usepackage{booktabs}
\usepackage{enumitem}
\setlist[itemize]{noitemsep}
\usepackage{abstract}
\renewcommand{\abstractnamefont}{\normalfont\bfseries}
\renewcommand{\abstracttextfont}{\normalfont\small\itshape}
\usepackage{titlesec}
\renewcommand\thesection{\Roman{section}}
\renewcommand\thesubsection{\Alph{subsection}}
\renewcommand\thesubsubsection{\arabic{subsubsection}}
\titleformat{\section}[block]{\normalsize\bfseries\scshape\centering}{\thesection.}{1em}{}
\titleformat{\subsection}[block]{\normalsize\bfseries\centering}{\thesubsection.}{1em}{}
\titleformat{\subsubsection}[block]{\normalsize\centering}{\thesubsubsection.}{1em}{}
\usepackage{fancyhdr}
\pagestyle{fancy}
\fancyhead{}
\fancyhead[C]{Computational Physics Homework $\bullet$ May 2018 $\bullet$ Vol. I, No. 2}
\usepackage{titling}
\usepackage{hyperref}
\hypersetup{unicode}
\AtBeginShipoutFirst{\input{zhwinfonts.tex}}
\usepackage{bm}
\usepackage{braket}
\usepackage{CJKutf8}
\usepackage{xcolor}
\usepackage{dcolumn}
\usepackage{graphicx}
\usepackage{indentfirst}
\usepackage{listings}
\usepackage[toc, page, title, titletoc, header]{appendix}
\definecolor{grey}{rgb}{0.8,0.8,0.8}
\definecolor{darkgreen}{rgb}{0,0.3,0}
\definecolor{darkblue}{rgb}{0,0,0.3}
\def\lstbasicfont{\fontfamily{pcr}\selectfont\footnotesize}
\lstset{
	numbers=left,
	numberstyle=\small,
	showstringspaces=false,
	showspaces=false,
	tabsize=4,
	frame=single,
	basicstyle={\footnotesize\lstbasicfont},
	keywordstyle=\color{darkblue}\bfseries,
	identifierstyle=,
	commentstyle=\color{darkgreen},
	stringstyle=\color{black}
}
\lstloadlanguages{C,C++,Fortran,Java,Matlab,Mathematica,Python}
\setlength{\parindent}{2em}
\begin{document}
\begin{CJK*}{UTF8}{gkai}
%----------------------------------------------------------------------------------------
%	TITLE SECTION
%----------------------------------------------------------------------------------------

\setlength{\droptitle}{-4\baselineskip} % Move the title up
\pretitle{\begin{center}\Huge\bfseries} % Article title formatting
	\posttitle{\end{center}} % Article title closing formatting
\title{计算物理第二次作业} % Article title
\author{
	\textsc{梁旭民}\thanks{\noindent 指导老师:齐新老师} \\[1ex] % Your name
	\normalsize Cuiying Hornors College, Lanzhou University \\ % Your institution
	\normalsize \href{mailto:liangxm15@lzu.edu.cn}{liangxm15@lzu.edu.cn} % Your email address
}
\date{}
\renewcommand{\maketitlehookd}{
	\begin{abstract}
		本次计算物理作业主要利用lorenz因子混沌方程的粒子学习了Euler法和Rungge-Kutta法求解常微分方程的初值问题,并尝试讨论行星绕恒星运动轨道及三体(两个恒星一个行星)的运行情况。
	\end{abstract}
}
\maketitle

%----------------------------------------------------------------------------------------
%	SECTION 1
%----------------------------------------------------------------------------------------

\section{Runge-Kutta法解常微分方程}
\subsection{问题描述}
推导3阶Runge-Kutta方程,并用推导的Runge-Kutta方程求解Lorenz吸引子方程组,画出蝴蝶
\begin{equation*}
	\left\{
	\begin{aligned}
		&\frac{dx}{dt}=10(y-x)\\
		&\frac{dy}{dt}=x(28-z)-y\\
		&\frac{dz}{dt}=xy--\frac{8}{3}z
	\end{aligned}
	\right.
\end{equation*}
初始条件为
\begin{equation*}
	\left\{
	\begin{aligned}
		&x(0)=5\\
		&y(0)=20\\
		&z(0)=-10
	\end{aligned}
	\right.
\end{equation*}
\subsection{推导3阶Runge-Kutta}
考虑下面的常微分方程初值问题
\begin{equation*}
	\left\{
	\begin{aligned}
		&\frac{dy}{dx}=f(x,y)\\
		&y(x_{0})=y_{0}
	\end{aligned}
	\right.
\end{equation*}
Rugge-Kutta的思想是在区间$[x_{n},x_{n+1}]$上取几个点的斜率,计算它们的平均斜率,从而求解出下一个相近邻位置处的函数值。对于3阶Rugge-Kutta来说,我们在区间$[x_{n},x_{n+1}]$区间上取三个点$x_{n},x_{n+p},x_{n+q}$的斜率值分别为$k_{1},k_{2},k_{3}$的加权平均作为平均斜率$k^{*}=\lambda_{1}k_{1}+\lambda_{2}k_{2}+\lambda_{3}k_{3}$,则有公式的形式为
\begin{equation*}
	y_{n+1}=y(x_{n})+h(\lambda_{1}k_{1}+\lambda_{2}k_{2}+\lambda_{3}k_{3})
\end{equation*}
其中
\begin{equation*}
	k_{1}=f(x_{n},y_{n})
\end{equation*}
取第二个点为$x_{n+p}$为
\begin{equation*}
	x_{n+p}=x_{n}+np\qquad 0<p<1
\end{equation*}
利用$k_{1}$来预报$y_{n+p}$,根据Euler格式有$x_{n+p}$处的$y_{n+p}$处的预报值
\begin{equation*}
	\tilde{y}_{n+p}=y_{n}+phk_{1}
\end{equation*}
根据原方程可以计算出$x_{n+p}$处的斜率值为
\begin{equation*}
	k_{2}=f(x_{n}+ph,y_{n}+phk_{1})
\end{equation*}
取第三个点为$x_{n+q}$为
\begin{equation*}
	x_{n+q}=x_{n}+qh\qquad 0<p<q<1
\end{equation*}
利用$k_{1},k_{2}$来预报$y_{n+q}$,根据Euler格式有$x_{n+q}$处的$y_{n+q}$处的预报值
\begin{equation*}
	\tilde{y}_{n+q}=y_{n}+qh(rk_{1}+sk_{2})
\end{equation*}
根据原方程可以计算出$x_{n+q}$处的斜率值为
\begin{equation*}
	k_{3}=f(x_{n+q},\tilde{y}_{n+q})=f(x_{n+q},y_{n}+qh(rk_{1}+sk_{2}))
\end{equation*}
因此整理可得
\begin{equation*}
	\left\{
	\begin{aligned}
		&k_{1}=f(x_{n},y_{n})\\
		&k_{2}=f(x_{n}+ph,y_{n}+phk_{1})\\
		&k_{3}=f(x_{n}+qh,y_{n}+qh(rk_{1}+sk_{2}))\\
		&y_{n+1}=y(x_{n})+h(\lambda_{1}k_{1}+\lambda_{2}k_{2}+\lambda_{3}k_{3})
	\end{aligned}
	\right.
\end{equation*}
注意到
\begin{equation*}
	\begin{aligned}
		&y^{\prime}(x_{n})=f(x_{n},y_{n})\\
		&y^{\prime\prime}(x_{n})=\frac{d}{dx}f(x_{n},y_{n}(x_{n}))\\
		& =f_{x}^{\prime}(x_{n},y_{n})+f_{y}^{\prime}(x_{n},y_{n})f(x_{n},y_{n})\\
		&y^{(3)}(x)=\frac{d}{dx}y^{\prime\prime}(x_{n})\\
		& =f_{xx}^{\prime\prime}(x_{n},y_{n})+2f(x_{n},y_{n})f_{xy}^{\prime\prime}(x_{n},y_{n}) +f^{2}(x_{n},y_{n})\\&f_{yy}^{\prime\prime}(x_{n},y_{n})+f_{y}^{\prime}(x_{n},y_{n})[f_{x}^{\prime}(x_{n},y_{n})+f(x_{n},y_{n})f_{y}^{\prime}(x_{n},y_{n})]
	\end{aligned}
\end{equation*}
因此则有
\begin{equation*}
	\left\{
	\begin{aligned}
		&k_{1}=f(x_{n},y_{n})=y^{\prime}(x_{n})\\
		&k_{2}=f(x_{n}+ph,y_{n}+phk_{1})\\
		&\quad =f(x_{n},y_{n})+phf_{x}^{\prime}(x_{n},y_{n})+phk_{1}f_{y}^{\prime}(x_{n},y_{n})\\
		&\quad +\frac{(ph)^{2}}{2}\left[f_{xx}^{\prime\prime}(x_{n},y_{n})+k_{1}^{2}f_{yy}^{\prime\prime}(x_{n},y_{n})\right]+O(h^{3})\\
		&k_{3}=f(x_{n}+qh,y_{n}+qh(rk_{1}+sk_{2}))\\
		&\quad =f(x_{n},y_{n})+qhf_{x}^{\prime}(x_{n},y_{n})+qh(rk_{1}+sk_{2})f_{y}^{\prime}(x_{n},y_{n})\\
		&\quad +\frac{(qh)^{2}}{2}[f_{xx}^{\prime\prime}(x_{n},y_{n})+(rk_{1}+sk_{2})^{2}f_{yy}^{\prime\prime}(x_{n},y_{n})]+O(h^{3})\\
		&y_{n+1}=y(x_{n})+h(\lambda_{1}k_{1}+\lambda_{2}k_{2}+\lambda_{3}k_{3})
	\end{aligned}
	\right.
\end{equation*}
与$y_{n+1}$在$x_{n}$处的Taylor展开式
\begin{equation*}
	y(x_{n+1})=y(x_{n})+hy^{\prime}(x_{n})+\frac{h^{2}}{2}y^{\prime\prime}(x_{n})+\frac{h^{3}}{6}y^{(3)}(x_{n})+O(h^{4})
\end{equation*}
相比较,可以得到以下不定方程组
\begin{equation*}
	\left\{
	\begin{aligned}
		&r+s=1\\
		&\lambda_{1}+\lambda_{2}+\lambda_{3}=1\\
		&\lambda_{2}p+\lambda_{3}q=\frac{1}{2}\\
		&\lambda_{2}p^{2}+\lambda_{3}q^{2}=\frac{1}{3}\\
		&\lambda_{3}pqs=\frac{1}{6}
	\end{aligned}
	\right.	
\end{equation*}
取$p=\frac{1}{2},q=1$,我们可以解得
\begin{equation*}
	\left\{
	\begin{aligned}
		&\lambda_{1}=\frac{1}{6}\\
		&\lambda_{2}=\frac{2}{3}\\
		&\lambda_{3}=\frac{1}{6}\\
		&r=-1\\
		&s=2
	\end{aligned}
	\right.
\end{equation*}
因此便可以得到我们常用的3阶Runge-Kutta方法
\begin{equation*}
	\left\{
	\begin{aligned}
		&k_{1}=f(x_{n},y_{n})\\
		&k_{2}=f(x_{x}+\frac{h}{2},y_{n}+\frac{h}{2})\\
		&k_{3}=f(x_{n}+h,y_{n}-hk_{1}+2hk_{2})\\
		&y_{n+1}=y_{n}+\frac{h}{6}(k_{1}+4k_{2}+k_{3})
	\end{aligned}
	\right.
\end{equation*}
\subsection{求解Lorenz吸引子方程}
我选择了0.001作为步长,并且求解了100000次,利用Python绘制成Figure 1的Lorenz蝴蝶图像。
\begin{figure}[h]
	\centering
	\includegraphics[width=0.9\linewidth]{./figure/Lorenz}
	\caption{Lorenz蝴蝶图像}
	\label{fig:Lorenz}
\end{figure}

%----------------------------------------------------------------------------------------
%	SECTION 2
%----------------------------------------------------------------------------------------

\section{地球公转与机械能}
\subsection{问题描述}
用你所知道的最高精度和最低精度算法计算地球公转轨道。并统计系统机械能。讨论高精度和低精度算法在保持机械能守恒的物理前提下,所消耗的计算资源。
\subsection{问题分析}
由于该问题是两体问题,且受到的是有心力,因此只需要考虑一个平面内的问题即可。设太阳的质量为M,地球的质量为m,约化质量为$m_{a}$,选择极坐标系研究该问题,设地球相对于太阳的距离为r,转过的角度为$\theta$,角动量为$L=m_{a}h$,根据角动量守恒及能量守恒
\begin{equation*}
	\left\{
	\begin{aligned}
		r^{2}m_{a}\frac{\partial \theta}{\partial t}&=L\\
		\frac{1}{m_{a}}(\left(\frac{\partial r}{\partial t}\right)^{2}+r^{2}\left(\frac{\partial \theta}{\partial t}\right)^{2})-\frac{GMm}{r}&=E
	\end{aligned}
	\right.
\end{equation*}
为了求解轨道方程我们消去时间t,则可以得到轨道的常微分方程
\begin{equation*}
	\frac{L}{m_{a}r^{2}\sqrt{\frac{2}{m_{a}}(E-\frac{1}{2}\frac{L^{2}}{m_{a}r^{2}}+\frac{GMm}{r})}}=d\theta
\end{equation*}
即得到r关于$\theta$的一阶常微分方程
\begin{equation*}
	\frac{dr}{d\theta}=\sqrt{ar^{4}+br^{3}-r^{2}}
\end{equation*}
其中
\begin{equation*}
	a=\frac{2E(m+M)}{mMh^{2}}\qquad b=\frac{2G(m+M)}{h^{2}}
\end{equation*}
\subsection{数据分析}
我在求解地球轨道的时候,分别选择了2、3、4阶Runge-kutta法,其中2阶Runge-kutta法相当于Eular法作为低精度算法,4阶Runge-kutta认为是高精度算法,分别求解出平均轨道$\bar{r}$,然后可以求解出地球公转的平均机械能
\begin{equation*}
	E=-\frac{2GMm}{r}
\end{equation*}
见Table 1。其中,轨道半径中位数为$r_{\frac{1}{2}}$,
公转轨道机械能表征值为$\bar{E}$。
\setlength{\tabcolsep}{0mm}{
\begin{table}[htbp]
	\centering
	\caption{地球公转轨道及机械能}
	\begin{tabular}{c|c|c|c}
		\hline
		Runge-Kutta阶数 & 2   & 3   & 4 \\
		\hline
		$r_{\frac{1}{2}}(\times 10^{11})$ & 1.510363 & 1.510363 & 1.510363\\
		$\bar{E}(\times 10^{34})$  & -1.048877 & -1.048877  & -1.048877 
	\end{tabular}
	\label{tab:addlabel}
\end{table}}

%----------------------------------------------------------------------------------------
%	SECTION 3
%----------------------------------------------------------------------------------------

\section{成为三体星人}
\subsection{问题描述}
如上题中,如果太阳系中加入第二颗恒星,导致地球在两颗恒星引力下运动,我们将成为典型的三体星人,讨论此情形下我们的公转轨道。

由于在求解18个方程的常微分方程组的时候对于两颗恒星参数调节以尝试讨论所有情况还没有完全完成,因此这里姑且现将该问题搁置,后续将会对于该问题有专门的讨论。

%----------------------------------------------------------------------------------------
%	APPENDICES SECTION
%----------------------------------------------------------------------------------------

\newpage
\onecolumn
\begin{appendices}
\section{Lorenz Equation}
Here is the program to solve the Lorenz ODE.\\
\textbf{\textcolor[rgb]{0.98,0.00,0.00}{Input C source:}}
\lstinputlisting[language=C]{./program/Lorenz.c}

\section{Two body question}
Here is the program to solve the Earth Revolution Orbit.\\
\textbf{\textcolor[rgb]{0.98,0.00,0.00}{Input C source:}}
\lstinputlisting[language=C]{./program/Runge-kutta(USSR).c}

\end{appendices}

%----------------------------------------------------------------------------------------
%	REFERENCE
%----------------------------------------------------------------------------------------

\newpage
\renewcommand\refname{参考文献}
\begin{thebibliography}{99}
\bibitem{ref1}Runge, Carl David Tolmé (1895), "Über die numerische Auflösung von Differentialgleichungen", Mathematische Annalen, Springer, 46 (2): 167–178, doi:10.1007/BF01446807.
\bibitem{ref2}Kutta, Martin Wilhelm (1901), Beitrag zur näherungweisen Integration totaler Differentialgleichungen.
\bibitem{ref3}Ascher, Uri M.; Petzold, Linda R. (1998), Computer Methods for Ordinary Differential Equations and Differential-Algebraic Equations, Philadelphia: Society for Industrial and Applied Mathematics, ISBN 978-0-89871-412-8.
\bibitem{ref4}Atkinson, Kendall A. (1989), An Introduction to Numerical Analysis (2nd ed.), New York: John Wiley \& Sons, ISBN 978-0-471-50023-0.
\bibitem{ref5}Butcher, John C. (May 1963), Coefficients for the study of Runge-Kutta integration processes, 3 (2), pp. 185–201, doi:10.1017/S1446788700027932.
\bibitem{ref6}Butcher, John C. (1975), "A stability property of implicit Runge-Kutta methods", BIT, 15: 358–361, doi:10.1007/bf01931672.
\bibitem{ref7}Butcher, John C. (2008), Numerical Methods for Ordinary Differential Equations, New York: John Wiley \& Sons, ISBN 978-0-470-72335-7.
\bibitem{ref8}Williams, David R. (2004-09-01). "Earth Fact Sheet". NASA. Retrieved 2007-03-17.
\end{thebibliography} 

%----------------------------------------------------------------------------------------
\end{CJK*}
\end{document}
